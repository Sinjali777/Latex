\documentclass{report}
\usepackage{times} % Required for inserting images
\usepackage{lipsum}
\usepackage{emoji}
\usepackage{array}
\usepackage{tabularx}
\usepackage{multirow}
\usepackage{longtable}
\usepackage{{graphicx}}
\usepackage{wrapfig}
\usepackage{natbib}
\usepackage{listings}
\usepackage{amsmath}
\usepackage{minted}
\usepackage{subcaption}
\usepackage[rightcaption]{sidecap}
\graphicspath{{./images/}} 
\renewcommand{\contentsname}{Table of Contents}
\renewcommand{\labelenumi}{\arabic{enumi}.\arabic{enumi}}
\renewcommand{\labelenumiii}
{\arabic{enumi}.arabic{enumi}.\arabic{enumiii}}
\begin{document}

\tableofcontents
\addcontentsline{toc}{section}{Name}

\chapter{INTRODUCTION}
%unnumbered section
\section*{Background}
\lipsum[1-2]
\section{motivation}
\lipsum[1-2]
\begin{itemize}
    \item item no 1
    \item item no 2
    \begin{enumerate}
        \item subitem 1
        \item subitem 2
    \end{enumerate}
\end{itemize}
\begin{enumerate}
\item one
    \item item no a
    \item item no b
    \item[!] item no c
    \item[0] emoji
\end{enumerate}
\begin{description}
    \item[hello, my name is sugam:] item no 1
    \item[b.] item no 2
\end{description}
    
\chapter{LITERATURE REVIEW}
\lipsum[1-2]
%New chapter
\section{Preview}
\lipsum[1-2]
    \par
    Table :

\begin{center}
    \begin{tabular}
    {|r|r|r|}
    \hline
        cell1 & cell2 & cell3 \\
        \hline
        cell4 & cell 5 & cell 6\\
        \hline
        cell7 & cell8 & cell 9 \\
    \hline
    \end{tabular}
\end{center}
\subsection{Subsection}
\lipsum[1-2]
\subsection{All about Tables}
\begin{center}
    \begin{tabular}{|c|c|c|}
    \hline
        \textbf{Fruit} & \textbf{Color} & \textbf{Average Weight(g)} \\
        \hline
        Apple & Red & 182 \\
        \hline
        Banana & Yellow & 118 \\
        \hline
        Cherry & Red & 8 \\
        \hline
        Orange & Orange & 131 \\
        \hline
        Grape & Purple & 5 \\
        \hline
    \end{tabular}
\end{center}
\begin{center}
    \begin{tabular}{|m{5em}|m{1cm}|m{1cm}|}
        \hline
        cell 1 & cell 2 & THis is for overflow \\
        \hline
        cell 1 & cell 2 & cell 3 \\
       \hline
       cell 1 & cell 2 & cell 3 \\
       \hline
    \end{tabular}
\end{center}

\subsection{Table covering the page width}
\begin{center}
    \begin{tabularx}{0.8\textwidth}{
        | >{\raggedright\arraybackslash}X
        | >{\centering\arraybackslash}X
        | >{\raggedleft\arraybackslash}X|}    
        \hline
        item 1 & item 2 & item 3 \\
        \hline
        item 1 & item 2 & item 3 \\
        \hline
        item 1 & item 2 & item 3 \\
        \hline
    \end{tabularx}   
\end{center}

\begin{center}
\subsection{Creating a full Functional Table}
\end{center}
    \begin{tabular}{|c|c|c|c|}
       \hline
       \multicolumn{4}{|c|}{Cell combine } \\
       \hline
        col 1 & col2 & col3 & col4  \\
       \hline
       \multirow{3}{*}{Multi row} & cell 2& cell 0 & cell 3 \\
       \cline{2-4}
       & cell 4 & cell 5 & cell 6 \\
       \cline{2-4}
       & cell 7 & cell 8 & cell 9 \\
       \hline
    \end{tabular}

\begin{center}
    \begin{tabular}{|c|c|c|c|}
         \hline
         \multirow{2}{*}{\textbf{Fruit}} & \multicolumn{3}{|c|}{\textbf{Sales(kg) }} \\
         \cline{2-4}
         & \textbf{North} & \textbf{South} & \textbf{Total} \\
         \hline
         Apple & 150 & 200 & 350  \\
         \hline
        \multirow{2}{*}{Banana} & 100 & 120 & \multirow{2}{*}{250} \\
        \cline{2-3}
        & 60 & 80 & \\
        \hline
        Cherry & 90 & 110 & 200 \\
        \hline
        Orange & 200 & 220 & 420 \\
        \hline
        \multirow{2}{*}{Grape} & 75 & 85 & \multirow{2}{*}{160} \\
        \cline{2-3}
        & 40 & 45 & \\
        \hline
        \end{tabular}
\end{center}    

\begin{center}
    \begin{tabular}{|c|c|c|c|c|}
    \hline
        \multirow{2}{*}{\textbf{Fruit}} & \multirow{2}{*}{\textbf{Region}} & \multicolumn{2}{|c|}{Quarter} & Total Sales(kg) \\
        \cline{3-4}
         & & Q1 & Q2 & \\
         \hline
         \multirow{3}{*}{Apple} & North & 150 & 150 & \multirow{3}{*}{510} \\
         \cline{2-4}
         & South & 120 & 130 & \\
         \cline{2-4}
         & East & 70 & 60 & \\
         \hline
         \multirow{3}{*}{Banana} & North & 200 & 220 & \multirow{3}{*}{620} \\
         \cline{2-4}
         & South & 180 & 160 & \\
         \cline{2-4}
         & East & 20 & 60 & \\
         \hline
         \multirow{3}{*}{Cherry} & North & 100 & 90 & \multirow{3}{*}{270} \\
         \cline{2-4}
         & South & 80 & 50 & \\
         \cline{2-4}
         & East & 40 & 30 & \\
         \hline
    \end{tabular}
\end{center}
\begin{figure}
    \centering
    \includegraphics[width=0.5\linewidth]{images/uffy.jpg}
    \caption{GOAT}
    \label{fig:enter-label}
\end{figure}
\subsection{How to use Long Table}
\begin{longtable}[c]{| c | c |}
\caption{Long table caption.\label{long}}\\
 \hline
 \multicolumn{2}{| c |}{Begin of Table}\\
 \hline
 Something & something else\\
 \hline
 \endfirsthead
 
 \hline
 \multicolumn{2}{|c|}{Continuation of Table \ref{long}}\\
 \hline
 Something & something else\\
 \hline
 \endhead
 
 \hline
 \endfoot
 
 \hline
 \multicolumn{2}{| c |}{End of Table}\\
 \hline\hline
 \endlastfoot
 
Lots of lines & like this\\
 Lots of lines & like this\\
 Lots of lines & like this\\
 Lots of lines & like this\\
 Lots of lines & like this\\
 Lots of lines & like this\\
 Lots of lines & like this\\
 Lots of lines & like this\\
 Lots of lines & like this\\
 Lots of lines & like this\\
 Lots of lines & like this\\
 Lots of lines & like this\\
 Lots of lines & like this\\
 Lots of lines & like this\\
 Lots of lines & like this\\
 Lots of lines & like this\\
 Lots of lines & like this\\
 Lots of lines & like this\\
 Lots of lines & like this\\
 Lots of lines & like this\\
 Lots of lines & like this\\
 Lots of lines & like this\\
 Lots of lines & like this\\
 Lots of lines & like this\\
 Lots of lines & like this\\
 Lots of lines & like this\\
 Lots of lines & like this\\
 Lots of lines & like this\\
 Lots of lines & like this\\
 Lots of lines & like this\\
 Lots of lines & like this\\
 Lots of lines & like this\\
 Lots of lines & like this\\
 Lots of lines & like this\\
 Lots of lines & like this\\
 Lots of lines & like this\\
 Lots of lines & like this\\
 Lots of lines & like this\\
 Lots of lines & like this\\
 Lots of lines & like this\\
 Lots of lines & like this\\
 Lots of lines & like this\\
 Lots of lines & like this\\
 Lots of lines & like this\\
 Lots of lines & like this\\
 Lots of lines & like this\\
 Lots of lines & like this\\
 Lots of lines & like this\\
 Lots of lines & like this\\
 Lots of lines & like this\\
 Lots of lines & like this\\
 Lots of lines & like this\\
 Lots of lines & like this\\
 Lots of lines & like this\\
 Lots of lines & like this\\
 Lots of lines & like this\\
 Lots of lines & like this\\
 Lots of lines & like this\\
 Lots of lines & like this\\
 Lots of lines & like this\\
 Lots of lines & like this\\
 Lots of lines & like this\\
 Lots of lines & like this\\
 Lots of lines & like this\\
 Lots of lines & like this\\
 Lots of lines & like this\\
 Lots of lines & like this\\
 Lots of lines & like this\\
 Lots of lines & like this\\
 Lots of lines & like this\\
 Lots of lines & like this\\
 Lots of lines & like this\\
 Lots of lines & like this\\
 Lots of lines & like this\\
 Lots of lines & like this\\
 Lots of lines & like this\\
 Lots of lines & like this\\
 Lots of lines & like this\\
 Lots of lines & like this\\
 Lots of lines & like this\\
 Lots of lines & like this\\
 Lots of lines & like this\\
 Lots of lines & like this\\
 Lots of lines & like this\\
 Lots of lines & like this\\
 Lots of lines & like this\\
 Lots of lines & like this\\
 Lots of lines & like this\\
 Lots of lines & like this\\
 Lots of lines & like this\\
 Lots of lines & like this\\
 Lots of lines & like this\\
 Lots of lines & like this\\
 Lots of lines & like this\\
 Lots of lines & like this\\
 Lots of lines & like this\\
 Lots of lines & like this\\
 Lots of lines & like this\\
 Lots of lines & like this\\
 Lots of lines & like this\\
 Lots of lines & like this\\
 Lots of lines & like this\\
 Lots of lines & like this\\
 Lots of lines & like this\\
 Lots of lines & like this\\
 Lots of lines & like this\\
 Lots of lines & like this\\
 Lots of lines & like this\\
 Lots of lines & like this\\
 Lots of lines & like this\\
 Lots of lines & like this\\
 Lots of lines & like this\\
 Lots of lines & like this\\
 Lots of lines & like this\\
 Lots of lines & like this\\
 Lots of lines & like this\\
 Lots of lines & like this\\
 Lots of lines & like this\\
 Lots of lines & like this\\
 Lots of lines & like this\\
 Lots of lines & like this\\
 Lots of lines & like this\\
 Lots of lines & like this\\
 Lots of lines & like this\\
 Lots of lines & like this\\
 Lots of lines & like this\\
 Lots of lines & like this\\
 Lots of lines & like this\\
 Lots of lines & like this\\
 Lots of lines & like this\\
 Lots of lines & like this\\
 Lots of lines & like this\\
 Lots of lines & like this\\
 Lots of lines & like this\\
 Lots of lines & like this\\
 Lots of lines & like this\\
 Lots of lines & like this\\
 Lots of lines & like this\\
 Lots of lines & like this\\
 Lots of lines & like this\\
 Lots of lines & like this\\
 Lots of lines & like this\\
 Lots of lines & like this\\
 Lots of lines & like this\\
 Lots of lines & like this\\
 Lots of lines & like this\\
 Lots of lines & like this\\
 Lots of lines & like this\\
 Lots of lines & like this\\
 Lots of lines & like this\\
 Lots of lines & like this\\
 Lots of lines & like this\\
 Lots of lines & like this\\
 Lots of lines & like this\\
 Lots of lines & like this\\
 Lots of lines & like this\\
 Lots of lines & like this\\
 Lots of lines & like this\\
 Lots of lines & like this\\
 Lots of lines & like this\\
 Lots of lines & like this\\
 Lots of lines & like this\\
 Lots of lines & like this\\
 Lots of lines & like this\\
 Lots of lines & like this\\
 Lots of lines & like this\\
 Lots of lines & like this\\
 Lots of lines & like this\\
 Lots of lines & like this\\
 Lots of lines & like this\\
 \end{longtable}
\listoftables
\vspace{10 pt}
Table \ref{tab:my_label} is an example of a reference
\begin{table}[h!]
    \centering
    \begin{tabular}{|c|c|c|}
        \hline
        col1 & col2 & col3  \\
        \hline
        1 & 2 & 3 \\
        4 & 5 & 6 \\
        7 & 8 & 9 \\
        \hline
    \end{tabular}
    \caption{Reference Table}
    \label{tab:my_label}
\end{table}

%%%%%%%%%%%%%%%%%%%%%%%%%%%%%%%%%%%%%%%%%%%%%%%%%%%#
% images
%%%%%%%%%%%%%%%%%%%%%%%%%%%%%%%%%%%%%%%%%%%%%%%%%
\section{Images}
\lipsum[1-2]
\includegraphics[width=0.6\textwidth]{one_piece}
\includegraphics[width=0.8 \textwidth]{luffy} \\
\includegraphics[width=0.8\textwidth, height = 10cm , angle= 180]{luffytaro}
\lipsum[1-2]
\begin{wrapfigure}{r}{0.25\textwidth}
    \centering  
    \includegraphics[width=0.25\textwidth]{images/luffy.jpg}
\end{wrapfigure}
\lipsum[1-2]
\begin{figure}[b] %bottom
    \centering
    \includegraphics[width=0.8\textwidth]{images/team.jpg}
    \caption{caption}
    \label{fig:enter-label}
\end{figure}
\lipsum[1-2]
\begin{figure}
    \centering
    \begin{subfigure}{0.4\textwidth}
       \includegraphics[width=0.8\textwidth]{images/luffy.jpg}
    \caption{Caption}
    \label{fig:enter-label} 
    \end{subfigure}
    \hfill
    \begin{subfigure}{0.4\textwidth}
        \includegraphics[width=0.8\textwidth]{images/uffy.jpg}
        \label{fig:first}
        \caption{first sub figure }
    \end{subfigure}
\caption{Creating subfigure in \LaTeX}
\label{fig:figures}
\end{figure}
\begin{SCfigure}[0.5][h]
    \caption{The picture of the temple, This caption will be on the right}
    \includegraphics[width=0.6 \textwidth]{images/zoro.jpg}
\end{SCfigure}
\begin{figure}[h!]
    \centering
    \includegraphics[width=0.6 \textwidth]{images/zoro.jpg}
    \caption{Caption}
    \label{fig:enter-label}
\end{figure}
\lipsum[1-2] \\
%%%%%%%%%%%%%%%%%%%%%%%%%%%%%%%%%%%%%%%%%%%%%%%%%
%Day 6- Inline math mode
\section{Inline Maths Mode}

\noindent Standard \LaTeX{} practice is to write inline math by enclosing it between \verb|\(...\)|:\\
Its the simple mathematical expression : \(x^2+y^2-z^2\)
\noindent Instead of write inline math by enclosing it between \verb|(...\):|\\ we can use \texttt(\$....\$)
Its the simple mathematical expression : $x^2+y^2-z^2$ \\
\noindent Or you can use 
\verb|\begin{math}...\end{math}|:
Its the simple mathematical expression:
\begin{math}x^2+y^2=z^^2\end{math}
\subsection{Display maths}
%% Display math.
Display math section
three different ways to write display math mode:
1. \verb|\[....\]| Expression : \[x^2+y^2\]
2. \verb|\begin{displaymath}....\end{displaymath}| expression: \begin{displaymath}x^2+y^2\end{displaymath}
3. \verb|\begin{equation}...\end{equation}| expression: \begin{equation}
    x^2+y^2=z 
\end{equation}
\subsection{Mathematical symbols}
$\alpha A \\
\beta B \\
\gamma \ \Gamma \ \delta \
\theta \ \infty 
\int \ \oint \ \sum \ \prod
$
\subsection{Subscript and Superscript}
1. We use \_\ for subscript.

$
a_2 \\
a_{ij+1}
$ \\
2. We use\^\ for superscript.\\
$
a^2
a^{ij-1}
$ \\
\\
$
 a_{ij+1}^{ij-1}\\
 \\
(a_1^2)^3 \\
$
\section{How to use Brackets}
We can use Brackets like this is \LaTeX. \\

$
\{x+y\} \\
\langle x+1 \rangle
$\\

\noindent Newton's Law of Gravitation
\[F=G(\frac{m_1m_2}{r^2})\] 

\noindent Using this we can use bigger brackets. \\
\[
F=G\left(\frac{m_1m_2}{r^2}\right)
\]
Multi-line Equations \\
\begin{align*}
      y=1+ &\left(\frac{1}{x}+ \frac{1}{x^2} + \frac{1}{x^3} + \ldots \right. \\
        &\left. \quad + \frac{1}{x^{n-1}} + \frac{1}{x^n} \right )
\end{align*}
\\
Binomial Expression
\[\binom{n}{k}=\frac{n!}{k!(n-k)!}\] \\
Taylor Series
\[f(x)=\sum_{n=0}^{\infty} \frac{f^{(n)}(a)}{n!}(x-a)^n\]
Euler's Theorem
\[e^{i \theta}=\cos \theta + i \sin \theta\]

\noindent Fourier Theorem
\[
\mathcal{F}\{f(t)\}=F{\omega}=\int_{-\infty}^{\infty} f(t) e^{-i \omega t}\,dt
\]
Schrodinger Equation
\[
i \hbar \frac{\partial}{\partial t} \Psi{\mathbf{r},t}= \left (-\frac{\hbar^2}{2m}c \nabla^2 + v(\mathbf{r},t) \right) 
\]

Maxwell's Equation
\begin{align*}   
& \nabla . \textbf{E} = \frac{\rho}{\epsilon_0}  \\
& \nabla . \textbf{B} = 0 \\
& \nabla x \textbf{E}= -\frac{\partial B}{\partial t} \\
& \nabla x B = \mu_0 J + \mu_0 \epsilon_0 \frac{\partial E}{\partial t} 
\end{align*}

%%%%%%%%%%%%%%%%%%%%%%%%%%%%%%%%%%%%%%%%%%%%%%%%%%%%%%%%%%%%%%%%%%

\section{Matrix}


\begin{enumerate}
    \item Plain: $ \begin{matrix} 1& 2 & 3 \\ a & b & c \end{matrix}$
    \item Parenthesis; round brackets : $\begin{pmatrix}
        1 & 2 & 3 \\ a & b & c 
    \end{pmatrix}$
    \item Braces; square brackets : $ \begin{bmatrix}
        1 & 2 & 3 \\ a & b & c
    \end{bmatrix}$
    \item  Braces; curly brackets : $\begin{Bmatrix}
       1 & 2 & 3 \\ a & b & c 
   \end{Bmatrix}$ 
    \item Pipes: $\begin{vmatrix}
        1 & 2 & 3 \\ a & b & c
    \end{vmatrix}$
    \item Double pipes: $ \begin{Vmatrix}
        1 & 2 & 3 \\ a & b & c
    \end{Vmatrix}$ \\
    $\begin{pmatrix}
        a & b \\
        c & d
    \end{pmatrix}$
    but it looks too big, so let's try
    $\big(\begin{smallmatrix}
        a & b \\
        c & d
    \end{smallmatrix}\big)$
    instead.

\[
c= \begin{bmatrix}
    \begin{vmatrix}
        1 & 4 \\
        6 & 0
    \end{vmatrix} & - \begin{vmatrix}
        1 & 2 \\
        3 & 4 \\
    \end{vmatrix} & \begin{vmatrix}
        5 & 6 \\
        7 & -7
    \end{vmatrix} \\
    - \begin{vmatrix}
        1 & 2 \\
        4 & 5
    \end{vmatrix} & \begin{vmatrix}
        1 & 3 \\ 
        4 & -8
    \end{vmatrix} & - \begin{vmatrix}
        1 & 2 \\
        3 & -5 
    \end{vmatrix} \\
     \begin{vmatrix}
        1 & 2 \\
        4 & 5
    \end{vmatrix} & - \begin{vmatrix}
        1 & 3 \\ 
        4 & -8
    \end{vmatrix} & \begin{vmatrix}
        1 & 2 \\
        3 & -5 
    \end{vmatrix} \\
\end{bmatrix}
\]
\end{enumerate}

\section{Size and Spacing with typeset mathematics}
Fraction typeset within a paragraph typically look like this: \(\frac{3x}{2}\). \\
For longer display style \(\displaystyle \frac{3x}{2}\) which also has an effect on line spacing. \\
\(\scriptstyle  \frac{3x}{2}\) or \(\scriptscriptstyle \frac{3x}{2}\) \\
 Equally, you can change the style of mathematics normally typeset in display style :

 \[f(x)=\frac{p(x)}{Q(x)}\quad \textrm{and} \quad \textstyle 
 f(x)=\frac{p(x)}{Q(x)}\quad \textrm{and} \quad \scriptstyle f(x)= \frac{P(x)}{Q(x)} 
 \]
   
 \[
  \cfrac[|]{1}{\sqrt{2} +
    \cfrac[r]{1}{\sqrt{2}+
    \cfrac{1}{\sqrt{2}+\dotsb }}}
 \]
%%%%%%%%%%%%%%%%%%%%%%%%%%%%%%%%%%%%%%%%%%%%%%
 \section{IEEE citation}

Uncertainity modeling and estimation are being increasingly used to deep learning based medical imaging application. \cite{bharati2020deep,jogin2018feature, nasteski2017overview}

Uncertainity modeling and estimation are being increasingly used to deep learning based medical imaging application. \cite{jogin2018feature,reich2009rich}

\bibliographystyle{chicago}
\bibliography{references}

%for report-Bibilography, for article- References
\lstlistoflistings
\newpage
\section{Code Listing}
\begin{lstlisting}[language=python, caption=Python example]
    import pandas as pd

    def load_df{load_path}:
        df = pd.read_csv{file_path}
        return df

    dataframe = load_df(file_path)
    print(dataframe.head(10))    
\end{lstlisting}

%%%%%%%%%%%%%%%%%%%%%%%%%%%%%%%%%%%%%%%%%%%%%%%%%%
%Use minted
\listoflistings
\newpage
\begin{listing}
    \begin{minted}{python}
        import pandas as pd

    def load_df{load_path}:
        df = pd.read_csv{file_path}
        return df

    dataframe = load_df(file_path)
    print(dataframe.head(10))
    \end{minted}
    \caption{Example python}
    \label{listing:1}
\end{listing}
\end{document}
